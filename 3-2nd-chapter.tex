\chapter{Разработка архитектуры}
\label{cha:ch_2}

Для начала декомпозируем задачу. Нужно реализовать:
\begin{enumerate}
    \item Кнопку меню;
    \item Строку местоположения;
    \item Счетчик количества открытых браузерных вкладок;
    \item Логотип, который может превращаться в дудл;
    \item Омнибокс с кнопкой камеры;
    \item Информеры погоды, дорожного трафика, почты и чатов;
    \item Вкладки сервисов.
\end{enumerate}

Теперь нам нужно разработать базовый макет, расположив все эти элементы на нем.

Для начала заметим, что некоторые элементы можно сгруппировать таким образом,
чтобы получившиеся группы были расположены вертикально друг за другом.
Такая группировка позволяет нам использовать в качестве корневого элемента макета
LinearLayout — контейнер, распологающий элементы подряд вертикально или горизонтально.

Сгруппированные вместе кнопку меню, строку местоположения
и счетчик количества браузерных вкладок поместим во вложенный LinearLayout,
расположив элементы горизонтально друг относительно друга.
С информерами поступим так же.

Описав такую верстку в xml-файле, мы получаем возможность во время исполнения приложения
долучить дерево объектов, каждый из которых унаследован от класса View.
Объекты класса View представляют из себя элементы, отвечающие за отображение данных пользователю.

Для каждого из элементов нужно будет написать свою логику обновления данных
и реагирования на события, порожденные пользователем.
Для этого заведем класс BenderViewHolder, с которым будем взаимодействовать,
который будет взаимодействовать с обертками над внутренними компонентами.
