\chapter{Реализация компонента}
\label{cha:ch_3}

\section{Кнопка меню}

Кнопка имеет два состояния: с красной точкой и без нее.

В верстке используем \texttt{ImageView}, предназначенную для отображения иконок.
В \texttt{BenderHamburgerViewHolder} поддерживаем два состояния:
с красной точкой и без нее устанавливаем иконку
в зависимости от состояния, сообщаемого снаружи.
Кроме того, позволяем снаружи подписаться на событие клика пользователем по \texttt{ImageView}.

\section{Строка статуса}

В верстке используем \texttt{TextView}, позволяющий отображать текст и иконки вокруг текста.
В \texttt{BenderStatusTextViewHolder} поддерживаем два состояния: онлайн и офлайн.
В онлайн-состоянии слева от текста проставляем иконку-стрелку,
текст меняем на название города, передаваемого при переходе в это состояние.
В~офлайн-состоянии проставляем иконку с изображением восклицательного знака
и текст \mbox{«Нет подключения к сети»}.
Позволяем снаружи подписаться на событие клика в онлайн-состоянии,
чтобы пользователь мог вручную изменить город.

\section{Счетчик количества браузерных вкладок}

В верстке используем \texttt{TextView} для отображения количества открытых вкладок.
В \texttt{BenderBrowserTabCounterViewHolder} поддерживаем изменение числа
и позволяем снаружи подписываться на событие клика.

\section{Логотип и дудл}

В верстке используем \texttt{ImageView}.
В \texttt{BenderLogoAndDoodleViewHolder} устанавливаем логотип по умолчанию
в качестве картинки, но позволяем снаружи менять ее, чтобы показывать дудл.
Позволяем снаружи подписываться на событие клика,
чтобы была возможность открыть страницу с информацией о дудле.

\section{Омнибокс}

В верстке для самого омнибокса используем контейнер \texttt{FrameLayout},
устанавливаем ему фон, внутрь кладем \texttt{ImageView} для кнопки камеры,
в качестве картинки для которой устанавливаем иконку камеры.
В \texttt{BenderOmniboxViewHolder} позволяем подписываться на события
клика по омнибоксу и по кнопке камеры.

\section{Информеры}

В верстке для информера используем \texttt{LinearLayout},
внутрь которого кладем \texttt{ImageView} для иконки и \texttt{TextView} для текста.
В \texttt{BenderInformerViewHolder} поддерживаем изменение иконки и текста,
позволяем подписываться на события клика.
Элементы верстки для каждого информера складываем в общий для них
контейнер \texttt{LinearLayout} с горизонтальной ориентацией.

После этого для каждого конкретного информера создаем свой класс,
в который помещаем логику установки иконки и текста.
Объектам этих классов делегируем логику через \texttt{BenderInformerContainerViewHolder}.

\section{Вкладки сервисов}

В верстке отдельной вкладки используем \texttt{LinearLayout},
внутрь него кладем \texttt{ImageView} для иконки и \texttt{TextView} для текста.
В \texttt{BenderTabViewHolder} поддерживаем два сотояния:
активное с~яркой иконкой и жирным текстом
и неактивное с~черно-белой иконкой и обычным текстом.
Также позволяем подписываться на события клика.

После этого вкладки нужно поместить в контейнер со специфической логикой.
Контейнер должен запрашивать у каждого элемента, который в нем лежит,
размер минимальной ширины, которую этот элемент должен занимать,
после этого проставлять такой размер всем элементам,
чтобы ширина каждого элемента была одинаковой.
При этом, если не все элементы помещаются на экран,
то контейнер должен скроллиться.
А если помещаются, то поместить их нужно в центре контейнера.

Готового такого контейнера не оказалось,
поэтому пришлось реализовывать свой.
Им стал \texttt{BenderTabLayoutSlidingStrip} — унаследованный
от \texttt{ViewGroup} — предка всех контейнеров — контейнер,
поддержавший всю описанную выше логику, за исключением скроллинга.
Он был помещен внутрь \texttt{HorizontalScrollView},
который, как следует из его названия, и поддерживает нужный скроллинг.
При~наследовании был переопределен метод \texttt{onMeasure}, вычисляющий
высоту и ширину контейнера по его элементам, а также метод \texttt{onLayout},
вычисляющий координаты для размещения каждого элемента.

Логика наполнения элементами \texttt{BenderTabLayoutSlidingStrip}
и управления их состояниями помещена в~\texttt{BenderTabLayoutViewHolder}.
Позволяем подписаться снаружи на клики по элементам
(при клике сообщаем индекс элемента, на который кликнули).
